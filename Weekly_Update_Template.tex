\documentclass[12pt]{article}
%%%%%%Preamble%%%%%%%%%%
\usepackage{scrtime} % for \thistime (this package MUST be listed first!)
\usepackage{fancyhdr}
\usepackage{xspace} %for having the perfect spacing after new commands
\usepackage{xcolor,colortbl}%for changing cell colour in tables
\usepackage{booktabs} %for tables
\usepackage{graphicx}
\usepackage{listings}
%\usepackage{Sweave}
%%%%formating the page%%%%
\pagestyle{fancy}

\usepackage{hyperref}
\hypersetup{colorlinks=true,linkcolor=blue,filecolor=magenta,urlcolor=cyan,}
\urlstyle{same}


\setlength{\headheight}{15.2pt}
\setlength{\headsep}{13 pt}
\setlength{\parindent}{28 pt}
\setlength{\parskip}{12 pt}
\pagestyle{fancyplain}
\usepackage[T1]{fontenc}
\rhead{\fancyplain{}{Pangenome Thesis $|$  \today \hfill Ann Le}} %insert your name here and your document title. the \today will just put the date that you compiled the documnet there.
\renewcommand\headrulewidth{0.5mm}
%%%%%%%%%%%%%%%%%%%%%%
%you can make new commands so you dont have to keep typing out (for example) the same bacteria name
%to use it in your text you just type \tub 
\newcommand{\salm}{\textit{Salmonella}\xspace
\newcommand{\saur}{\textit{S.\,aureus}\xspace}
\newcommand{\bas}{\textit{Bacillus subtilis}\xspace}
\newcommand{\strep}{\textit{Streptomyces}\xspace}
\newcommand{\ecol}{\textit{E.\,coli}\xspace}
\providecommand{\e}[1]{\ensuremath{\times 10^{#1}}}
%%%%%%%%%%%%%%%%%%%%%%%
\begin{document}


\bigskip	
	
\section*{Two Weeks Ago}

\begin{itemize} 

\item Mostly file organization and getting the github ready and started. Bacterial 
files were being tracked and the remaining blastp runs should be completed over the
time that I was away. 

\end{itemize}

\section*{This Week}
Obtained first indelmiss data on sample of five salmonella after reading through the 
documentation and trying to figure out how to work it.

This is the results that were shown:



\section*{Next Week}
Plan the final R data frames output:

%\begin{itemize}

%\end{itemize}

Columns: Bacteria Name -> Pangenome Sizes (Pan, Core etc.), Distance, Indelmiss (M1, M2, M3, M4)
run1
run2
.
.
.
run100

\begin{enumerate}

\item Reference roaryinput lists: 100 group runs of 20 species each 

\end{enumerate}

\end{document}
